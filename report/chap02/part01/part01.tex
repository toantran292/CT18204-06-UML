\phantomsection
\subsection{Các sơ đồ use case trong hệ thống}
\setcounter{subsubsection}{0}
\setcounter{figure}{0}

% \phantomsection
\subsubsection{Sơ đồ tổng quát}
Sơ đồ use case tổng quát của hệ thống bao gồm các nhóm người dùng (actors)
và các chức năng cơ bản của hệ thống được thể hiện ở \myref{fig:c2p1general}

% \begin{figure}[H]
%   \centering
%   \includesvg[inkscapelatex=false, width=16cm]{../uml/useCase/UseCaseTongQuat.svg}
%   \caption{\bfseries Sơ đồ use case tổng quát}
%   \label{fig:c2p1general}
% \end{figure}


\phantomsection
\subsubsection{Sơ đồ tổng quát}
Sơ đồ use case tổng quát của hệ thống bao gồm các nhóm người dùng
(actors) và các chức năng cơ bản của hệ thống được thể hiện ở \myref{fig:uc-1}

\begin{figure}[H]
  \centering
  \includesvg[inkscapelatex=false, width=\linewidth]{../uml/useCase/UseCaseTongQuat.svg}
  \caption{Sơ đồ use case tổng quát}
  \label{fig:uc-1}
\end{figure}

\phantomsection
\subsubsection{Chức năng của actor "Người dùng"}
Người dùng là nhóm người dùng chưa đăng ký tài khoản hoặc chưa đăng nhập vào hệ thống nhưng có nhu cầu xem thông tin các giải đấu, trận đấu.
Những chức năng của actor được mô tả như
\myref{fig:uc-2}

\begin{figure}[H]
  \centering
  \includesvg[inkscapelatex=false, width=\linewidth]{../uml/useCase/UseCaseNguoiDung.svg}
  \caption{Sơ đồ use case người dùng}
  \label{fig:uc-2}
\end{figure}

\noindent
\textbf{Chức năng xem diễn biến}

Người dùng có thể xem diễn biến của trận đấu hoặc giải đấu đang được diễn ra mà
không cần đăng nhập.

\noindent
\textbf{Chức năng xem thông tin chi tiết}

Cho phép người dùng có thể tra cứu các thông tin chi tiết về vận động viên,
đội thi đấu, đoàn thi đấu và giải đấu ngoài ra trong giải đấu người dùng còn có
thể xem lịch thi đấu và có quyền in lịch thi đấu mà không cần đăng nhập.

\noindent
\textbf{Chức năng đăng nhập}

Cho phép người dùng đăng nhập vào hệ thống bằng tài khoản đã đăng ký hoặc được cấp.

\noindent
\textbf{Chức năng đăng ký}

Cho phép người dùng đăng ký vào hệ thống.

\phantomsection
\subsubsection{Chức năng của actor "Người đã đăng nhập"}
Người đã đăng nhập bao gồm nhóm người dùng đã đăng ký tài khoản hoặc được cấp tài khoản và đã đăng nhập vào hệ thống.
Những chức năng của actor được mô tả như
\myref{fig:uc-3}

\begin{figure}[H]
  \centering
  \includesvg[inkscapelatex=false, width=\linewidth]{../uml/useCase/UseCaseNguoiDaDangNhap.svg}
  \caption{Sơ đồ use case người đã đăng nhập}
  \label{fig:uc-3}
\end{figure}

\noindent
\textbf{Chức năng đăng xuất}

Cho phép người đã đăng nhập đăng xuất khỏi hệ thống

\noindent
\textbf{Chức năng xem thông tin cá nhân}

Cho phép người đã đăng nhập xem thông tin tài khoản cá nhân.

\noindent
\textbf{Chức năng xem thông báo}

Cho phép người đã đăng nhập xem các thông báo mới về các trận đấu hoặc giải đấu đang theo dõi

\noindent
\textbf{Chức năng theo dõi trận đấu giải đấu}

Cho phép người đã đăng nhập theo dõi hoặc hủy theo dõi trận đấu, giải đấu sắp và đang diễn ra.



\phantomsection
\subsubsection{Chức năng của actor "Vận động viên"}
Người đã đăng nhập bao gồm nhóm người dùng đã đăng ký tài khoản hoặc được cấp tài khoản và đã đăng nhập vào hệ thống.
Những chức năng của actor được mô tả như
\myref{fig:uc-4}

\begin{figure}[H]
  \centering
  \includesvg[inkscapelatex=false, width=\linewidth]{../uml/useCase/UseCaseVanDongVien.svg}
  \caption{Sơ đồ use case vận động viên}
  \label{fig:uc-4}
\end{figure}


\noindent
\textbf{Chức năng gửi yêu cầu chỉnh sửa thông tin cá nhân}

Cho phép Vận động viên gửi yêu cầu chỉnh sửa các thông tin chưa đúng với thật tế.

\phantomsection
\subsubsection{Chức năng của actor "Huấn luyện viên"}
Huấn luyện viên là nhóm người dùng có chức năng quản lý một đội thi đấu, khiếu nại về kết quả của trận đấu mà đội huấn luyện viên dẫn tham gia.
Những chức năng của actor này được mô tả cụ thể như hình
\myref{fig:uc-5}.

\begin{figure}[H]
  \centering
  \includesvg[inkscapelatex=false, width=\linewidth]{../uml/useCase/UseCaseTruongBanToChuc.svg}
  \caption{Sơ đồ use case huấn luyện viên}
  \label{fig:uc-5}
\end{figure}

\noindent
\textbf{Chức năng duyệt yêu cầu:}

Cho phép huấn luyện viên duyệt yêu cầu và chỉnh sửa thông tin của vận động viên

\noindent
\textbf{Chức năng quản lý đội:}

Cho phép huấn luyện viên có thể quản lý đội.

\noindent
\textbf{Chức năng quản lý đội hình:}

Chọn đội hình thi đấu: Chọn đội hình từ đội hình mẫu thi đấu 1 trận đấu.\par
Thiết lập vai trò vận động viên: Khi xếp đội hình huấn luyện viên thiết lập via trò cho tung vận động viên trong đội hình.\par
Xếp đội hình: Huấn luyện viên có thể xếp đội hình mẫu để tham gia trận đấu


\noindent
\textbf{Chức năng gửi đơn khiếu nại:}

Cho phép huấn luyện viên có thể gửi đơn khiếu nại cho trường ban tổ chức.


\phantomsection
\subsubsection{Chức năng của actor "Trưởng đoàn"}
Trưởng đoàn là người chịu trách nhiệm quản lý tất cả thành viên trong đội, tài chính của đội ,và đăng kí giải đấu cho đội.
Những chức năng của actor được mô tả như hình
\myref{fig:uc-6}

\begin{figure}[H]
  \centering
  \includesvg[inkscapelatex=false, width=\linewidth]{../uml/useCase/UseCaseTruongBanToChuc.svg}
  \caption{Sơ đồ use case huấn luyện viên}
  \label{fig:uc-6}
\end{figure}


\phantomsection
\subsubsection{Chức năng của actor "Trưởng ban tổ chức"}

Trưởng ban tổ chức là người chịu trách nhiệm chung trong việc lập kế hoạch, điều hành và quản lý quá trình tổ chức của một giải đấu thể thao.
Những chức năng của actor này được mô tả cụ thể như hình
\myref{fig:uc-7}.
\begin{figure}[H]
  \centering
  \includesvg[inkscapelatex=false, width=\linewidth]{../uml/useCase/UseCaseTruongBanToChuc.svg}
  \caption{Sơ đồ use case trưởng ban tổ chức}
  \label{fig:uc-7}
\end{figure}

\noindent
\textbf{Chức năng quản lý giải đấu:}

Chức năng lập lịch giải đấu: Cho phép Trưởng ban tổ chức tạo lịch trình cho toàn bộ các trận đấu trong giải. Điều này bao gồm chọn ngày, giờ, và địa điểm cho mỗi trận đấu sao cho hợp lý và thuận tiện cho các đội tham gia.\par
Chức năng tạo giải đấu: Cho phép Trưởng ban tổ chức tạo và thiết lập các thông tin cơ bản về giải đấu như tên, ngày bắt đầu, số lượng đội tham gia, và hình thức thi đấu,...\par
Chức năng chỉnh sửa thông tin giải đấu: Cho phép Trưởng ban tổ chỉnh sửa các thông tin cơ bản về giải đấu như tên, ngày bắt đầu, số lượng đội tham gia, và hình thức thi đấu,...

\noindent
\textbf{Chức năng quản lý trận đấu:}

Chức năng tạo trận đấu: Cho phép trưởng ban tổ chức tạo ra và thiết lập lịch trình cho mỗi trận đấu, bao gồm các thông tin như đội tham gia, thời gian, địa điểm,...\par
Chức năng cập nhật diễn biến trận đấu: Cho phép trưởng ban tổ chức theo dõi và cập nhật diễn biến của mỗi trận đấu đang diễn ra. Bao gồm ghi nhận các bàn thắng, thẻ phạt,...

\noindent
\textbf{Chức năng duyệt đơn khiếu nại:}

Trưởng ban tổ chức có trách nhiệm tiếp nhận và duyệt đơn khiếu nại này. Sau đó, họ phải xem xét và ra quyết định xử lý mỗi khiếu nại một cách công bằng và minh bạch, đảm bảo sự công bằng và tính chuyên nghiệp trong quá trình tổ chức giải đấu.


\phantomsection
\subsubsection{Chức năng của actor "Quản trị viên"}

\phantomsection
\subsubsection{Chức năng của actor "Hệ thống QLGTDTD"}
