\phantomsection
\subsubsection{Use case "Tạo giải đấu"}
\setcounter{figure}{0}

\begin{longtblr}[caption = {Đặc tả usecase Tạo giải đấu},
  label = {tab:usecase1-spec},]{colspec={|l|p{.7\linewidth}|}}
  \hline
  \textbf{Tên usecase} & \textbf{Tạo giải đấu}                                                                        \\\hline
  Tóm tắt              & Cho phép Trưởng ban tổ chức tạo một giải đấu mới.                                            \\\hline
  Actor                & Trưởng ban tổ chức.                                                                          \\\hline
  Phiên bản            & 1.1                                                                                          \\\hline
  Chịu trách nghiệm    & Hà Ngọc Linh                                                                                 \\\hline
  Ngày tạo             & 13/03/2024                                                                                   \\\hline
  Ngày cập nhật        & 19/03/2024                                                                                   \\\hline
  Điều kiện tiên quyết & Trưởng ban tổ chức được cấp tài khoản từ Quản trị viên và đăng nhập thành công vào hệ thống. \\\hline
  Kịch bản thường      &
  \begin{minipage}{\linewidth}
    \vskip 4pt
    \begin{enumerate}
      \item Trưởng ban tổ chức sau khi đăng nhập vào hệ thống thì chọn danh mục Quản lý giải đấu.
      \item Hệ thống sẽ chuyển hướng đến giao diện Quản lý giải đấu.
      \item Trưởng ban tổ chức chọn danh mục Tạo giải đấu.
      \item Hệ thống sẽ chuyển hướng đến giao diện Tạo giải đấu.
      \item Trưởng ban tổ chức điền các thông tin để tạo giải đấu mới như: tên, ngày tổ chức, cơ cấu giải thưởng,...  \\
            \textbf{Có thể nhảy đến}\\
            \textbf{\textcolor{red}{A1}} -- Trưởng ban tổ chức không nhập nữa và nhấn vào nút thoát.
      \item Sau khi chỉnh sửa xong chọn tạo giải đấu, hệ thống sẽ kiểm tra tính hợp lệ.\\
            \textbf{Có thể nhảy đến}\\
            \textbf{\textcolor{red}{A2}} -- Thông tin không hợp lệ.
      \item Hệ thống tiến hành lưu thay đổi.Thông báo giải đấu được tạo thành công.
      \item Trở về giao diện Quản lý giải đấu.
    \end{enumerate}
    \vskip 1pt
  \end{minipage}
  \\\hline
  Kịch bản thay thế    &
  \begin{minipage}{\linewidth}
    \vskip 4pt
    \begin{enumerate}[label={\textbf{\textcolor{red}{A\arabic*}} --}, align=left, itemsep=-5pt]
      \item  Nhấn vào nút thoát.  \\
            \vspace{-1.5em}
            \begin{enumerate}[leftmargin=-5px, align=left, label=\arabic*.]
              \setcounter{enumii}{5}
              \item[]
                    \hspace{-25px} Chuỗi A1 bắt đầu ở bước 5 của kịch bản thường.
              \item Thông báo đã thoát giao diện tạo giải đấu.
              \item[]
                    \hspace{-25px} Trở về bước 2 của kịch bản thường.
            \end{enumerate}
      \item Thông tin không hợp lệ. \\
            \vspace{-1.5em}
            \begin{enumerate}[leftmargin=-5px, align=left, label=\arabic*.]
              \setcounter{enumii}{6}
              \item[]
                    \hspace{-25px} Chuỗi A2 bắt đầu ở bước 6 của kịch bản thường.
              \item Thông báo lỗi nhập liệu cụ thể cho Trưởng ban tổ chức.
              \item[]
                    \hspace{-25px} Trở về bước 5 của kịch bản thường.
            \end{enumerate}
    \end{enumerate}
    \vskip 1pt
  \end{minipage}
  \\\hline
  Kịch bản lỗi         &                                                                                              \\\hline
  Kết quả              & Tạo giải đấu mới thành công.                                                                 \\\hline
\end{longtblr}
