\phantomsection
\setsection{Chương 6: Tổng kết}
\setcounter{section}{6}

\subsection{Kết quả đạt được}
\setcounter{subsubsection}{0}
\subsubsection{Về kiến thức}
Sau khi hoàn thành đề tài "Hệ thống quản lý các giải đấu thể thao chuyên nghiệp", nhóm đã đạt được những kiến thức quan trọng sau:
\begin{itemize}[label={--}]
  \item Thiết kế hệ thống sử dụng UML: Nhóm đã tiếp cận và áp dụng thành thạo các nguyên lý và quy tắc thiết kế trong UML. Điều này giúp chúng tôi xây dựng các mô hình hệ thống chính xác và dễ hiểu, từ sơ đồ use case đến sơ đồ lớp và sơ đồ tuần tự, sơ đồ hoạt động.
  \item Mô tả chi tiết về cấu trúc và hoạt động của hệ thống: Nhóm đã có cái nhìn tổng quan và chi tiết về cách thức hệ thống hoạt động, bao gồm cả tương tác giữa các thành phần khác nhau. Điều này giúp chúng tôi hiểu rõ hơn về cách hệ thống được tổ chức và thực thi.
  \item Lập trình hướng đối tượng và ngôn ngữ mô hình hóa: Chúng tôi đã củng cố kiến thức về lập trình hướng đối tượng thông qua việc xây dựng ngôn ngữ mô hình hóa. Điều này giúp chúng tôi áp dụng các nguyên tắc lập trình hướng đối tượng vào dự án của mình một cách linh hoạt và chính xác.
  \item Sử dụng công cụ thiết kế dựa trên UML: Nhóm đã nắm vững việc sử dụng các công cụ thiết kế dựa trên UML như StarUML để tạo ra các sơ đồ rõ ràng và chuyên nghiệp, từ đó hỗ trợ quá trình phát triển và triển khai hệ thống một cách hiệu quả.
  \item Áp dụng Công nghệ thông tin trong quản lý giải đấu thể thao: Nhóm đã nắm vững quy trình áp dụng Công nghệ thông tin để tối ưu hóa quản lý các sự kiện thể thao chuyên nghiệp, từ việc lên kế hoạch đến tổ chức và thực thi các hoạt động
\end{itemize}

\subsubsection{Về kỹ năng}
Ngoài việc tiếp thu kiến thức, nhóm cũng đã phát triển những kỹ năng quan trọng như:
\begin{itemize}[label={--}]
  \item Làm việc nhóm và phân chia công việc: Nhóm đã rèn luyện khả năng làm việc nhóm, phân chia công việc một cách hiệu quả và linh hoạt. Việc này giúp chúng tôi tối ưu hóa sự hợp tác và đạt được kết quả tốt nhất trong quá trình làm dự án.
  \item Giao tiếp và trình bày: Chúng tôi đã phát triển khả năng giao tiếp và trình bày ý tưởng một cách rõ ràng và hiệu quả. Việc giao tiếp và trao đổi ý tưởng giữa các thành viên diễn hết sức thuận lợi và đi đến những thống nhất chung giữa các thành viên.
  \item Viết báo cáo chuyên nghiệp: Nhóm đã học cách viết báo cáo một cách chuyên nghiệp và khoa học dưới sự hướng dẫn của giảng viên. Việc này giúp chúng tôi trình bày kết quả của dự án một cách rõ ràng và logic, đồng thời cung cấp thông tin cần thiết cho các bên liên quan.
  \item Phân tích và đặc tả hệ thống: Chúng tôi đã phát triển khả năng phân tích và đặc tả hệ thống thông qua việc sử dụng ngôn ngữ mô hình hóa. Điều này giúp chúng tôi hiểu rõ hơn về cách thức hoạt động của hệ thống và cung cấp hướng dẫn chi tiết cho quá trình phát triển và triển khai.
  \item Sử dụng công cụ thiết kế UML: Nhóm đã nắm vững việc sử dụng các công cụ thiết kế starUML để tạo ra các sơ đồ và mô hình hóa hệ thống một cách chuyên nghiệp và hiệu quả. Điều này giúp chúng tôi tăng cường khả năng thực thi và triển khai dự án một cách hiệu quả.
\end{itemize}

\subsection{Hạn chế}
Ngoài những kết quả tích cực, nhóm cũng ghi nhận một số hạn chế như:
\begin{itemize}[label={--}]
  \item Chưa vận dụng được tối đa các kiến thức và tính năng về các sơ đồ đã học.
  \item Còn nhiều hạn chế trong việc định hình và thiết kế sơ đồ.
  \item thể thiết kế và mô tả tất cả các đối tượng và chức năng trên hệ thống một cách đầy đủ.
\end{itemize}

