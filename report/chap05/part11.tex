\phantomsection
\subsection{Sơ đồ tuần tự "Xem diễn biến trận đấu"}
\setcounter{figure}{0}

Chức năng "Xem diễn biến trận đấu" là một trong các chức năng của actor "".
Chức năng này đã được thể hiện ở \myref{}.
Sơ đồ tuần tự của chức năng này được thể hiện cụ thể ở \myref{} sau:

Người thiết kế: Trần Thái Toàn

Sơ đồ này đã được đặc tả cục thể tại mô tả Use case
"Xem diễn biến trận đấu" ở \myreftb{tab:usecase11-spec}

\noindent
\textbf{Mô tả:}\\
Người dùng sau khi đã truy cập vào hệ thống thì chọn danh mục "Xem diễn biến trận đấu loại" ở giao diện chính. Hệ thống sẽ chuyển hướng đến giao diện trang diễn biến trận đấu loại. \par
Hệ thống tiến hành lấy danh sách các trận đấu loại đang diễn ra từ CSDL và hiển thị lên giao diện người dùng. \par
Nếu người dùng chọn chức năng "Tìm kiếm" thì giao diện sẽ hiển thị thanh tìm kiếm. Khi người dùng nhập từ khóa vào thanh tìm kiếm và nhấn tìm kiếm thì hệ thống sẽ lấy danh sách các trận đấu loại đang diễn ra theo từ khóa mà người dùng đã nhập và hiển thị lên giao diện. Trong trường hợp không tìm thấy trận đấu đang diễn ra nào theo từ khóa mà người dùng đã nhập thì giao diện sẽ hiển thị thông báo không tìm thấy và hiển thị lại trang diễn biến trận đấu loại trước đó. \par
Khi người dùng chọn vào một trận đấu loại cụ thể thì hệ thống sẽ lấy diễn biến trận đấu từ CSDL và hiển thị lên giao diện cho người dùng.

\noindent
\textbf{Kết quả:} Người dùng có thể xem được diễn biến trận đấu


