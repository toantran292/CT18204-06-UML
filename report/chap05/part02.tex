\phantomsection
\subsection{Sơ đồ hoạt động "Cập nhật diễn biến trận đấu"}
\setcounter{figure}{0}

Chức năng "Cập nhật diễn biến trận đấu" là một trong các chức năng của actor "Trưởng ban tổ chức".
Chức năng này đã được thể hiện ở \myref{}.
Sơ đồ hoạt động của chức năng này được thể hiện cụ thể ở \myref{fig:at-2} sau:

\begin{figure}[H]
  \centering
  \includesvg[inkscapelatex=false, width=\linewidth]{../uml/activity/ADCapNhatDienBienTranDau.svg}
  \caption{Sơ đồ use case người dùng}
  \label{fig:at-2}
\end{figure}

Người thiết kế: Hà Ngọc Linh - B2207536

Sơ đồ này đã được đặc tả cục thể tại mô tả Use case "Cập nhật diễn biến trận đấu"
ở \myreftb{tab:usecase1-spec}

\noindent
\textbf{Mô tả:}\\
Trưởng ban tổ chức sau khi đã đăng nhập thành công vào hệ thống thì chọn danh mục “Quản lý giải đấu” ở giao diện chính. Hệ thống sẽ chuyển hướng đến giao diện Quản lý giải đấu.\par
Trưởng ban tổ chức chọn danh mục "Tạo giải đấu" ở giao diện Quản lý giải đấu. Hệ thống sẽ chuyển hướng đến giao diện Tạo trận đấu.\par
Trưởng ban tổ chức tiến hành nhập các thông tin cần thiết cho giải đấu.\par
Nếu Trưởng ban tổ chức không muốn Tạo giải đấu nữa và nhấn vào nút thoát thì hệ thống sẽ thoát khỏi chức năng Tạo giải đấu và chuyển hướng về giao diện Quản lý giải đấu. Nếu không thì vẫn tiếp tục nhập liệu như bình thường.\par
Sau khi đã kiểm tra lại thông tin, Trưởng ban tổ chức chọn vào nút Tạo giải đấu để thực hiện việc Tạo giải đấu mới. Hệ thống sẽ kiểm tra thông tin được nhập của giải đấu.\par
Nếu thông tin Trưởng ban tổ chức nhập không hợp lệ thì trả về thông báo thông tin không hợp lệ và quay trở lại giao diện Tạo trận đấu để Trưởng ban tổ chức tiến hành nhập lại thông tin.\par
Nếu thông tin Trưởng ban tổ chức nhập là hợp lệ thì tiến hành ghi nhận việc Tạo giải đấu lên hệ thống và trả về thông báo việc đăng ký thành công.

\noindent
\textbf{Kết quả:} Thao tác Tạo giải đấu mới thành công.


