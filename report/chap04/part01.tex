\phantomsection
\subsection{Sơ đồ tuần tự "Tạo giải đấu"}
\setcounter{figure}{0}

Chức năng "Tạo giải đấu" là một trong các chức năng của actor "Trưởng ban tổ chức". Chức năng này đã được thể hiện ở \myref{}. Sơ đồ tuần tự của chức năng này được thể hiện cụ thể ở \myref{} sau:

Người thiết kế: Hà Ngọc Linh B2207536
Sơ đồ này đã được đặc tả cục thể tại mô tả Use case "Tạo giải đấu" ở \myreftb{}

\noindent
\textbf{Mô tả chức năng:} Cho phép Trưởng ban tổ chức tạo ra giải đấu mới.

\noindent
\textbf{Điều kiện tiên quyết:} Trưởng ban tổ chức được cấp tài khoản từ Quản trị viên và đăng nhập thành công vào hệ thống.

\noindent
\textbf{Trình tự thực hiện:}
\begin{enumerate}
      \item Trưởng ban tổ chức sau khi đăng nhập vào hệ thống thì chọn danh mục Quản lý giải đấu.
      \item Hệ thống sẽ chuyển hướng đến giao diện Quản lý giải đấu.
      \item Trưởng ban tổ chức chọn danh mục Tạo giải đấu.
      \item Hệ thống sẽ chuyển hướng đến giao diện Tạo giải đấu.\\
            \textbf{[Loop]}
      \item    Trưởng ban tổ chức điền các thông tin để tạo giải đấu mới như: tên, ngày tổ chức, cơ cấu giải thưởng,...
      \item Người dùng nhấn vào nút thoát. \textbf{[Tùy chọn 1]}
      \item Hệ thống chuyển hướng đến giao diện Tạo giải đấu. Thoát khỏi vòng lặp. \\
            \textbf{[Kết thúc tùy chọn 1]}
      \item Sau khi chỉnh sửa xong chọn Tạo giải đấu.
      \item Hệ thống gọi phương thức kiemTraThongTin() để kiểm tra tính hợp lệ của thông tin vừa nhập.
      \item Trả về kết quả kiểm tra. \\
            \textbf{[Rẽ nhánh]}
      \item Nếu thông tin nhập không hợp lệ [ketQua == false]. Thông báo lỗi nhập liệu cụ thể cho Trưởng ban tổ chức. Trở về giao diện Tạo giải đấu, tiếp tục vòng lặp. \textbf{[Rẽ nhánh 1]}
      \item Ngược lại, nếu thông tin nhập hợp lệ [ketQua == true]. Hệ thống tiến hành lưu thay đổi. \textbf{[Rẽ nhánh 2]}
      \item Hệ thống trả về thông báo Tạo giải đấu thành công.
      \item Giao diện thông báo cho Trưởng ban tổ chức việc Tạo giải đấu đã thành công. Thoát khỏi vòng lặp.
      \item Trở về giao diện Quản lý giải đấu. \\
            \textbf{Kết Thúc}
\end{enumerate}

\noindent
\textbf{Kết quả:} Thao tác Tạo giải đấu mới thành công.

