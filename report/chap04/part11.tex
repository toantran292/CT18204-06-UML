\phantomsection
\subsection{Sơ đồ tuần tự "Xem diễn biến trận đấu"}
\setcounter{figure}{0}

Chức năng "Xem diễn biến trận đấu" là một trong các chức năng của actor "".
Chức năng này đã được thể hiện ở \myref{}.
Sơ đồ tuần tự của chức năng này được thể hiện cụ thể ở \myref{} sau:

Người thiết kế: Trần Thái Toàn

Sơ đồ này đã được đặc tả cục thể tại mô tả Use case
"Xem diễn biến trận đấu" ở \myreftb{tab:usecase11-spec}

\noindent
\textbf{Mô tả chức năng:} Cho phép người dùng có thể theo dõi diễn biến của một trận đấu loại đang được diễn ra.

\noindent
\textbf{Điều kiện tiên quyết:} Không có.

\noindent
\textbf{Trình tự thực hiện:}

\noindent
\begin{enumerate}
  \item Sau khi truy cập vào hệ thống, người dùng chọn chức năng "Xem diễn biến trận đấu"
  \item Hệ thống sẽ gọi phương thức layDSTranDauLoaiDienRa để lấy danh sách trận đấu loại đang diễn ra.
  \item CSDL trả về danh sách trận đấu loại đang diễn ra.
  \item Giao diện hiển thị danh sách trận đấu loại đang diễn ra.
  \item Người dùng chọn chức năng tìm kiếm. \textbf{[Tùy chọn 1]}
  \item Giao diện hiển thị trang tìm kiếm
  \item Người dùng nhập từ khóa tìm kiếm
  \item Người dùng nhấn nút tìm kiếm
  \item Hệ thống sẽ gọi phương thức layDSTranDauLoaiDienRa để lấy danh sách trận đấu loại đang diễn ra theo từ khóa người dùng cung cấp
  \item CSDL trả về danh sách trận đấu loại đang diễn ra theo từ khóa người dùng cung cấp. \textbf{[Rẽ nhánh]}
        \\\textbf{[Rẽ nhánh 1]}
  \item Nếu kết quả tìm thấy ít nhất một trận đấu loại đang diên ra thì giao diện hiển thị danh sách trận đấu loại đang diễn ra tìm được.
        \\\textbf{[Rẽ nhánh 2]}
  \item Nếu kết quả không tìm thấy trận đấu loại đang diễn ra thì hiển thị thông báo không tìm thấy trận đấu.
  \item Giao diện hiển thị danh sách trước đó.
        \\\textbf{[Kết thúc Tùy chọn 1]}
  \item Người dùng chọn vào một trận đấu đang diễn ra cụ thể để xem diễn biến.
  \item Hệ thống sẽ gọi phương thức layDienBien để lấy diễn biến trận đấu
  \item CSDL trả về chi tiết diễn biến
  \item Giao diễn hiển thị diễn biến của trận đấu.
        \\\textbf{[Kết thúc]}
\end{enumerate}

\noindent
\textbf{Kết quả:} Hiển thị diễn biến của trận đấu.

