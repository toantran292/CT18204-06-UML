\phantomsection
\subsection{Sơ đồ tuần tự "Xếp đội hình"}
\setcounter{figure}{0}

Chức năng "Xếp đội hình" là một trong các chức năng của actor "".
Chức năng này đã được thể hiện ở \myref{}.
Sơ đồ tuần tự của chức năng này được thể hiện cụ thể ở \myref{} sau:

Người thiết kế: Trần Minh Trí

Sơ đồ này đã được đặc tả cục thể tại mô tả Use case
"Xếp đội hình" ở \myreftb{tab:usecase9-spec}

\noindent
\textbf{Mô tả chức năng:} Cho phép huấn luyện viên có thể xếp đội hình mẫu để tham gia trận đấu.

\noindent
\textbf{Điều kiện tiên quyết:} Huấn luyện viên đã đăng nhập vào hệ thống.

\noindent
\textbf{Trình tự thực hiện:}

\noindent
\begin{enumerate}
      \item Sau khi Huấn luyện viên đăng nhập vào hệ thống ,Huấn luyện viên chọn chức năng quản lý đội hình.
      \item Hệ thống chuyển hướng đến giao diện quản lý đội hình.
      \item Huấn luyện viên chọn xếp đội hình.
            \\\textbf{[Loop]}
      \item Hệ thống chuyển hướng đến giao diện xếp đội hình.
            \\\textbf{[Lựu chọn]}
      \item Huấn luyện viên bấm nút thoát.
      \item Hệ thống chuyển về giao diện xếp đội hình và kêt thúc vòng lặp \textbf{[Loop]}.
      \item Gọi hàm layDSVDV() để lấy danh sách vận động viên gán và biến dsvdv.
      \item Trả về biến dsvdv.
      \item Chọn vận động viên
      \item Chọn vai trò cho vận động viên
      \item Huấn luyện viên bấm nút xong.
      \item Gọi hàm thietLapVaiTro với tham số id của vận động viên và tên vai trò để thiết lập vai trò cho vận động viên.
      \item Gửi thông báo xếp đội hình thành công và tiếp tục vòng lặp quay về \textbf{[Loop]}.
            \\\textbf{[Kết thúc]}
\end{enumerate}

\noindent
\textbf{Kết quả:}

