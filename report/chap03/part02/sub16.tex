\phantomsection
\subsubsection{Lớp TinhThanhPho}
\setcounter{figure}{0}
\setcounter{paragraph}{0}

\phantomsection
\paragraph{Mô tả thuộc tính của lớp TinhThanhPho}\mbox{}

Các thuộc tính của lớp TinhThanhPho được mô tả như ở \myreftb{tab:class16-1-spec}

\begin{adjustwidth}{-2cm}{-1cm}
  \begin{longtblr}[caption = {Mô tả thuộc tính của lớp TinhThanhPho},
    label = {tab:class16-1-spec},]{
    width=1\linewidth, rowhead=1, hlines,vlines,
    colspec={X[2,c]X[1,c]X[1,c]X[1,c]X[1,c]X[1,c]X[1,c]X[3,c]},
    rows={1.5cm,m},
    row{1}={font=\bfseries,c}}
    Tên thuộc tính & Kiểu truy cập & Kiểu dữ liệu & Giá trị mặc nhiên & Kích thước & Min & Max & Diễn giải             \\
    id             & public        & String       & Null              & 8          &     &     & id của đội hình mẫu.  \\
    tenDoiHinh     & public        & String       & Null              & 50         &     &     & Tên của đội hình mẫu. \\
  \end{longtblr}
\end{adjustwidth}

\phantomsection
\paragraph{Mô tả phương thức của lớp TinhThanhPho}\mbox{}

Các phương thức của lớp TinhThanhPho được mô tả như ở \myreftb{tab:class16-2-spec}

\begin{adjustwidth}{-2cm}{-1cm}
  \begin{longtblr}[caption = {Mô tả phương thức của lớp TinhThanhPho},
    label = {tab:class16-2-spec},]{
    width=1\linewidth, rowhead=1, hlines,vlines,
    colspec={X[3,c]X[1,c]X[3,c]X[1,c]X[1,c]X[1,c]X[2,c]X[3,c]},
    rows={1.5cm,m},
    row{1}={font=\bfseries,c}}
    Tên phương thức              & Kiểu truy cập          & Danh sách các tham số        & Kiểu dữ liệu tham số & Giá trị mặc nhiên & Kích thước & Kiểu trả về của phương thức & Diễn giải                                                                               \\
    \SetCell[r=3]{} xepDoiHinh() & \SetCell[r=3]{} public & \SetCell[c=4]{} Có 2 tham số &                      &                   &            & \SetCell[r=3]{} boolean     & \SetCell[r=3]{} Xếp mộ đội hình mẫu mới. Trả về true nếu thành công false nếu thất bại. \\
                                 &                        & tenDoiHinh                   & String               & Null              & 50         &                             &                                                                                         \\
                                 &                        & dsVDV                        & List                 & Null              &            &                             &                                                                                         \\
  \end{longtblr}
\end{adjustwidth}
