\phantomsection
\subsubsection{Lớp TruongBanToChuc}
\setcounter{figure}{0}
\setcounter{paragraph}{0}

\phantomsection
\paragraph{Mô tả phương thức của lớp TruongBanToChuc}\mbox{}

Các phương thức của lớp TruongBanToChuc được mô tả như ở \myreftb{tab:class6-2-spec}

\begin{adjustwidth}{-2cm}{-1cm}
  \begin{longtblr}[caption = {Mô tả phương thức của lớp TruongBanToChuc},
    label = {tab:class6-2-spec},]{
    width=1\linewidth, rowhead=1, hlines,vlines,
    colspec={X[3,c]X[1,c]X[3,c]X[1,c]X[1,c]X[1,c]X[2,c]X[3,c]},
    rows={1.5cm,m},
    row{1}={font=\bfseries,c}}
    Tên phương thức                & Kiểu truy cập          & Danh sách các tham số            & Kiểu dữ liệu tham số & Giá trị mặc nhiên & Kích thước & Kiểu trả về của phương thức & Diễn giải                                                                                \\
    \SetCell[r=2]{} taoTruongBTC() & \SetCell[r=2]{} public & \SetCell[c=4]{} Có 1 tham số     &                      &                   &            & \SetCell[r=2]{} boolean     & \SetCell[r=2]{} Thêm Trưởng ban tổ chức. Trả về true nếu thành công, false nếu thất bại. \\
                                   &                        & idTruongBTC                      & String               & Null              & 8          &                             &                                                                                          \\
    xoaTruongBTC()                 & public                 & \SetCell[c=4]{} Không có tham số &                      &                   &            & boolean                     & Xóa Trưởng ban tổ chức. Trả về true nếu thành công, false nếu thất bại.                  \\
  \end{longtblr}
\end{adjustwidth}
