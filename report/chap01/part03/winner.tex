\phantomsection
\subsubsection{Biểu mẫu tạo giải đấu}

Đây là biểu mẫu cho phép tạo giải đấu của hệ thống của Winner.
Giao diện được hiển thị như \myref{fig:winner-1}.
\figmini{images/winner-1.png}{fig:winner-1}{Giao diện biểu mẫu tạo giải đấu của hệ thống Winner}

\noindent
\textbf{Phân tích thành phần dữ liệu:}
\begin{itemize}[leftmargin=1.5cm, label={--}]
  \item Tên giải đấu.
  \item Logo giải đấu.
  \item Bộ môn thi đấu: Bóng đá, bóng rổ, bida...
  \item Hình thức thi đấu: cá nhân hoặc đội/cặp.

\end{itemize}

\phantomsection
\subsubsection{Biểu mẫu thiết lập thông tin giải đấu}

Đây là biểu mẫu cho phép thiết lập thông tin giải đấu của hệ thống của Winner.
Giao diện được hiển thị như \myref{fig:winner-2}.
\figmini{images/winner-2.png}{fig:winner-2}{Giao diện biểu mẫu thiết lập thông tin giải đấu của hệ thống Winner}

\noindent
\textbf{Phân tích thành phần dữ liệu:}
\begin{itemize}[leftmargin=1.5cm, label={--}]
  \item Tên bảng đấu
  \item Số lần đối đầu của mỗi đội trong bảng đấu
  \item Cách tính điểm của mỗi trận đấu gồm: Điểm chiến thắng, điểm hòa, điểm thua
  \item Tiêu chí xếp hạng của bảng đấu: Tính điểm tổng cộng hoặc tính từ một trận đấu trực tiếp
  \item Số cầu thủ thắng hạng/xuống hạng
  \item Điểm thưởng/điểm phạt
  \item Chế độ đấu:
        \begin{itemize}[label={+}]
          \item Bình thường: Mỗi cầu thủ chỉ đấu với những cầu thủ trong cùng một bảng
          \item Hỗn hợp: Mỗi cầu thủ đấu với tất cả những cầu thủ khác

        \end{itemize}

\end{itemize}
