\phantomsection
\subsection{Nhóm người dùng chính của hệ thống}
\setcounter{subsection}{4}
\setcounter{figure}{0}

\begin{itemize}[leftmargin=1.5cm, label={\textbf{--}}]
  \item \textbf{Trưởng ban tổ chức:} Các ban tổ chức giải đấu.
  \item \textbf{Trưởng đoàn:} Các đoàn thi đấu thể thao.
  \item \textbf{HLV:} Huấn luyện viên của các đội thi đấu.
  \item \textbf{VDV:} Vận động viên tham gia vào các trận đấu.
  \item \textbf{Người đã đăng nhập:} Những người đã có tài khoản trên hệ thống sau khi đăng ký hoặc được cấp có nhu cầu nhận thông báo về các diễn biến của trận đấu, giải đấu.
  \item \textbf{Người dùng:} Những người không có tài khoản, những người không có nhu cầu nhận thông báo về các diễn biến trận đấu.
  \item \textbf{Quản trị viên:} Người chịu trách nhiệm quản lý, cập nhật, sửa lỗi, bảo trì hệ thống.
\end{itemize}

\phantomsection
\subsubsection{Trưởng ban tổ chức}
\noindent
\textbf{Mô tả}

Người chịu trách nhiệm chung trong việc lập kế hoạch,
điều hành và quản lý quá trình tổ chức của một giải đấu thể thao.
Sử dụng phần mềm để tạo giải đấu, trận đấu,
lập lịch thi đấu và quản lý các khía cạnh khác của giải đấu như tài chính,
thống kê, ...

\noindent
\textbf{Quyền và trách nhiệm}
\begin{itemize}[leftmargin=1.5cm, label={\textbf{--}}]
  \item Có thể tạo và chỉnh sửa thông tin giải đấu.
  \item Có thể lập lịch thi đấu.
  \item Có thể quản lý thông tin của các đoàn, đội và vận động viên tham gia thi đấu.
  \item Có trách nhiệm về mọi khía cạnh của giải đấu như: xác định lịch thi đấu, địa điểm, quy định, tài chính, an ninh,...
  \item Phải đảm bảo rằng tất cả các đội và vận động viên tham gia giải đấu được đối xử công bằng và tuân thủ các quy định cạnh tranh.
\end{itemize}

\phantomsection
\subsubsection{Trưởng đoàn}
\noindent
\textbf{Mô tả}

Người quản lý các đội tuyển. Sử dụng phần mềm để đăng ký tham gia giải đấu,
quản lý thông tin về các đội tuyển và quản lý vận động viên trong đoàn,
theo dõi thông tin về các giải đấu. Được cấp tài khoản bởi quản trị viên.

\noindent
\textbf{Quyền và trách nhiệm}
\begin{itemize}[leftmargin=1.5cm, label={\textbf{--}}]
  \item Chỉnh sửa thông tin vận động viên trong đoàn thi đấu mà mình quản lý.
  \item Có thể lựa chọn đăng ký tham gia các giải đấu.
  \item Có thể cập nhật các thông tin của các đội tuyển trong đoàn thi đấu của mình, giữ cho thông tin các đội tuyển luôn được chính xác, minh bạch.
\end{itemize}

\phantomsection
\subsubsection{HLV}
\noindent
\textbf{Mô tả}

Người chịu trách nhiệm quản lý một đội thi đấu và các vận động viên trong đội đó
như quản lý đội hình thi đấu, thông tin của vận động viên và
có thể gửi đơn khiếu nại về kết quả thi đấu của đội.

\noindent
\textbf{Quyền và trách nhiệm}
\begin{itemize}[leftmargin=1.5cm, label={\textbf{--}}]
  \item Có quyền chỉnh sửa thông tin vận động viên trong đội thi đấu mà mình quản lý.
  \item Quản lý đội hình thi đấu,chọn đội hình thi đấu.
  \item Phải đảm bảo tính chính xác của thông tin vận động viên trong đội
  \item Tổ chức các đội hình hợp lệ, hiệu quả.
\end{itemize}


\phantomsection
\subsubsection{VĐV}
\noindent
\textbf{Mô tả}

Người trực tiếp tham gia thi đấu trong các trận đáu, giải thi đấu thể thao.

\noindent
\textbf{Quyền và trách nhiệm}
\begin{itemize}[leftmargin=1.5cm, label={\textbf{--}}]
  \item Có quyền xem và yêu cầu chỉnh sửa thông tin cá nhân
  \item Có trách nhiệm về tính chính xác thông tin cá nhân của mình.
\end{itemize}

\phantomsection
\subsubsection{Người đã đăng nhập}
\noindent
\textbf{Mô tả}

Những người đã có tài khoản trên hệ thống sau khi đăng ký hoặc
được cấp có nhu cầu nhận thông báo và theo dõi về các diễn biến của trận đấu,
giải đấu, đội tuyển yêu thích.

\noindent
\textbf{Quyền và trách nhiệm}
\begin{itemize}[leftmargin=1.5cm, label={\textbf{--}}]
  \item Đăng nhập vào hệ thống và tra cứu các thông tin trên hệ thống.
  \item Nhận thông báo,thông tin về đội bóng yêu thích.
  \item Theo dõi chi tiết  tiến trình giải đấu, lịch thi đấu đội bóng mình yêu thích.
  \item Xem thông tin cá nhân.
\end{itemize}

\phantomsection
\subsubsection{Người dùng}
\noindent
\textbf{Mô tả}

Người dùng chưa có tài khoản hoặc chưa đăng nhập vào hệ thống hoặc chưa có tài khoản
có nhu cầu tìm hiểu các thông tin về giải đấu, trận đấu đang diễn ra.

\noindent
\textbf{Quyền và trách nhiệm}
\begin{itemize}[leftmargin=1.5cm, label={\textbf{--}}]
  \item Xem thông tin các trận đấu, giải đấu.
  \item Xem lịch thi đấu, kết quả của các trận đấu, giải đấu.
  \item Có thể đăng ký, đăng nhập tài khoản cá nhân.
\end{itemize}

\phantomsection
\subsubsection{Quản trị viên}
\noindent
\textbf{Mô tả}

Quản trị viên có quyền cao nhất trong hệ thống,
quản lý toàn bộ phần mềm, quản lý tài khoản người dùng và
giám sát toàn bộ hệ thống để đảm bảo tính an toàn và ổn định của hệ thống.

\noindent
\textbf{Quyền và trách nhiệm}
\begin{itemize}[leftmargin=1.5cm, label={\textbf{--}}]
  \item Tạo và chỉnh sửa dữ liệu sân vận động trong hệ thống, giữ các dữ liệu về sân vận động luôn chính xác và cập nhật kịp thời.
  \item Tạo và chỉnh sửa dữ liệu hạng mục thi đấu trong hệ thống, giữ cho các dữ liệu hạng mục luôn được bổ sung và cập nhật dựa trên xu hướng tổ chức giải đấu.
  \item Quản lý hệ thống, đảm bảo hệ thống hoạt động ổn định và khắc phục các sự cố gây ra lỗi hệ thống.
  \item Xem các thống kê về số lượng người dùng truy cập và thống kê về tỷ lệ được chọn tổ chức thi đấu của các môn thể thao để có thể đánh giá hiệu suất của hệ thống và xu hướng tổ chức giải thi đấu.
  \item Tạo và vô hiệu hóa tài khoản người dùng, phân quyền người dùng, đảm bảo quyền hạn của các người dùng khác nhau và giữ cho thông tin tài khoản luôn được bảo mật
  \item Tạo và chỉnh sửa dữ liệu về môn thể thao, giữ cho các môn thể thao phù hợp và bắt kịp xu hướng tổ chức thể thao.
  \item Tạo và chỉnh sửa dữ liệu về nhà tài trợ, giữ các dữ liệu về nhà tài trợ luôn chính xác.
  \item Tạo và chỉnh sửa dữ liệu về đoàn thi đấu, giữ cho các dữ liệu về đoàn thi đấu chính xác và bổ xung các đoàn thi đấu mới.
\end{itemize}
