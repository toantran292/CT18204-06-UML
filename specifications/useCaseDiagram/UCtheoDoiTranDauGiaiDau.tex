\documentclass{article}
\usepackage[paperheight=29.7cm,paperwidth=21cm,right=2cm,left=1cm,top=2cm,bottom=2.5cm]{geometry}
\usepackage[vietnamese]{babel}
\usepackage[utf8]{inputenc}
\usepackage{array, multirow}
\usepackage{xcolor}
\usepackage{longtable}
\usepackage{lipsum} 
\usepackage{enumitem}
\usepackage{tabularray}
\renewcommand{\arraystretch}{1.5} % Cell height scaling

\DefTblrTemplate{contfoot-text}{default}{Tiếp tục bên trang tiếp theo}
\DefTblrTemplate{conthead-text}{default}{(Tiếp tục)}

\begin{document}

% useCase

\begin{longtblr}[caption = {Đặc tả usecase 	Theo dõi trận đấu, giải đấu},
  label = {tab:usecase1-spec},]{colspec={|l|p{.7\linewidth}|}}
  \hline
  \textbf{Tên usecase} & \textbf{Theo dõi trận đấu, giải đấu}                                                                        \\\hline
  Tóm tắt              & Cho phép Người đã đăng nhập theo dõi trận đấu, giải đấu mà mình quan tâm.                                            \\\hline
  Actor                & Người đã đăng nhập.                                                                          \\\hline
  Phiên bản            & 1.0                                                                                          \\\hline
  Chịu trách nghiệm    & Khưu Thị Bích Ngọc                                                                                 \\\hline
  Ngày tạo             & 17/04/2024                                                                                   \\\hline
  Ngày cập nhật        &                                                                                    \\\hline
  Điều kiện tiên quyết & Người đã đăng nhập đã đăng nhập thành công vào hệ thống và đang xem thông tin chi tiết trân đấu hoặc giải đấu nào đó. \\\hline
  Kịch bản thường      &
  \begin{minipage}{\linewidth}
    \vskip 4pt
    \begin{enumerate}
      \item  Người đã đăng nhập ấn chọn nút theo dõi của trận đấu, giải đấu.
      \item  Giao diện gửi yêu cầu kiểm tra xem người dùng đã theo dõi trận đấu, giải đấu đó hay chưa.
      \item  Hệ thống tiến hành kiểm tra.
      \item Hệ thống trả về kết quả vừa kiểm tra.  \\
            \textbf{Có thể nhảy đến}\\
            \textbf{\textcolor{red}{A1}} -- Người dùng đã theo dõi.
      \item Nếu người dùng chưa theo dõi, giao diện hiển thị để người dùng chọn cấp độ theo dõi mong muốn.
      \item Người dùng chọn cấp độ theo dõi và ấn gửi.
      \item Hệ thống lưu thông tin theo dõi của người.  \\
            \textbf{Có thể nhảy đến}\\
            \textbf{\textcolor{red}{A2}} -- Lưu thông tin theo dõi thất bại.
       \item Nếu lưu thông tin theo dõi thành công, hệ thống gửi thông báo thành công về trình duyệt.
       \item Trình duyệt hiển thị kết quả.
    \end{enumerate}
    \vskip 1pt
  \end{minipage}
  \\\hline
  Kịch bản thay thế    &
  \begin{minipage}{\linewidth}
    \vskip 4pt
    \begin{enumerate}[label={\textbf{\textcolor{red}{A\arabic*}} --}, align=left, itemsep=-5pt]
      \item  Người dùng đã theo dõi.  \\
            \vspace{-1.5em}
            \begin{enumerate}[leftmargin=-5px, align=left, label=\arabic*.]
              \setcounter{enumii}{4}
              \item[]
                    \hspace{-25px} Chuỗi A1 bắt đầu ở bước 4 của kịch bản thường.
              \item Trình duyệt yêu cầu người dùng xác nhận tiếp tục hay dừng quá trình hủy theo dõi.\\
                  \textbf{Có thể nhảy đến}\\
                  \textbf{\textcolor{red}{A3}} -- Người dùng dừng quá trình hủy yêu cầu.
              \item Người dùng chọn tiếp tục thực hiện.
              \item Hệ thống hủy theo dõi trận đấu, giải đấu của người dùng đó.\\
                  \textbf{Có thể nhảy đến}\\
                  \textbf{\textcolor{red}{A4}} -- Hủy theo dỗi thất bại.
            
              \item Nếu hủy theo dõi thành công, hệ thống gửi thông báo thành công về giao diện.
              \item[]
                    \hspace{-25px} Trở về bước 9 của kịch bản thường.
            \end{enumerate}
            
       \item Lưu thông tin theo dõi thất bại. \\
            \vspace{-1.5em}
            \begin{enumerate}[leftmargin=-5px, align=left, label=\arabic*.]
              \setcounter{enumii}{7}
              \item[]
                    \hspace{-25px} 	Chuỗi A2 bắt đầu ở bước 7 của kịch bản thường.
              \item Hệ thống gửi thông báo tạo theo dõi thất bại.
              \item[]
                    \hspace{-25px} Trở về bước 9 của kịch bản thường.
            \end{enumerate}
            
         \item Người dùng dừng quá trình hủy yêu cầu. \\
            \vspace{-1.5em}
            \begin{enumerate}[leftmargin=-5px, align=left, label=\arabic*.]
              \setcounter{enumii}{5}
              \item[]
                    \hspace{-25px} 	Chuỗi A3 bắt đầu ở bước 5 của kịch bản thay thế.
              \item Trở về giao diện ban đầu và kết thúc kịch bản.
            \end{enumerate}

        \item Huỷ theo dõi thất bại. \\
            \vspace{-1.5em}
            \begin{enumerate}[leftmargin=-5px, align=left, label=\arabic*.]
              \setcounter{enumii}{7}
              \item[]
                    \hspace{-25px} 	Chuỗi A4 bắt đầu ở bước 7 của kịch bản thay thế.
              \item Hệ thống gửi thông báo hủy theo dõi thất bại về giao diện.
              \item[]
                    \hspace{-25px} Trở về bước 9 của kịch bản thường.
            \end{enumerate}
            
    \end{enumerate}
    \vskip 1pt
  \end{minipage}
  \\\hline
  Kịch bản lỗi         &                                                                                              \\\hline
  Kết quả              & Thao tác theo dõi/hủy theo dõi trận đấu, giải đấu thành công.                                                                 \\\hline
\end{longtblr}
\end{document}