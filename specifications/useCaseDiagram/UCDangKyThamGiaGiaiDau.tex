\documentclass{article}
\usepackage[paperheight=29.7cm,paperwidth=21cm,right=2cm,left=1cm,top=2cm,bottom=2.5cm]{geometry}
\usepackage[vietnamese]{babel}
\usepackage[utf8]{inputenc}
\usepackage{array, multirow}
\usepackage{xcolor}
\usepackage{longtable}
\usepackage{lipsum} 
\usepackage{enumitem}
\usepackage{tabularray}
\renewcommand{\arraystretch}{1.5} % Cell height scaling

\DefTblrTemplate{contfoot-text}{default}{Tiếp tục bên trang tiếp theo}
\DefTblrTemplate{conthead-text}{default}{(Tiếp tục)}

\begin{document}

% useCase

\begin{longtblr}[caption = {Đặc tả usecase Đăng ký tham gia giải đấu},
  label = {tab:usecase1-spec},]{colspec={|l|p{.7\linewidth}|}}
  \hline
  \textbf{Tên usecase} & \textbf{Đăng kí tham gia giải đấu}                                                                        \\\hline
  Tóm tắt              & Cho phép Trưởng đoàn đăng ký tham gia giải đấu.                                            \\\hline
  Actor                & Trưởng đoàn.                                                                          \\\hline
  Phiên bản            & 1.1                                                                                          \\\hline
  Chịu trách nghiệm    & Nguyễn Tuấn Đạt                                                                                 \\\hline
  Ngày tạo             & 13/03/2024                                                                                   \\\hline
  Ngày cập nhật        & 19/03/2024                                                                                   \\\hline
  Điều kiện tiên quyết & Đã đăng nhập vào hệ thống bằng tài khoản của trưởng đoàn. \\\hline
  Kịch bản thường      &
  \begin{minipage}{\linewidth}
    \vskip 4pt
    \begin{enumerate}
      \item Sau khi trưởng đoàn đăng nhập vào hệ thống, trưởng đoàn chọn vào danh mục danh sách giải đấu.
      \item Hệ thống chuyển sang giao diện danh sách giải đấu.
      \item Trưởng đoàn chọn giải đấu cần đăng ký tham gia.
      \item Hệ thống chuyển sang giao diện chi tiết giải đấu.      
      \item Trưởng đoàn chọn vào chức năng đăng ký tham gia giải đấu  \\
            \textbf{Có thể nhảy đến}\\
            \textbf{\textcolor{red}{A1}} -- Trưởng đoàn không đăng ký nữa và nhấn nút thoát.
      \item Trưởng đoàn chọn hạng mục tham gia thi đấu và các đội thi đấu ở các hạng mục đã chọn.
      \item Trưởng đoàn nhấn vào nút xác nhận đăng ký tham gia.\\
            \textbf{Có thể nhảy đến}\\
            \textbf{\textcolor{red}{A2}} -- Không thể đăng ký tham gia giải đấu đã chọn.
      \item Hệ thống thông báo đăng ký tham gia giải đấu thành công.
      \item Trở về giao diện chi tiết giải đấu trước đó.
    \end{enumerate}
    \vskip 1pt
  \end{minipage}
  \\\hline
  Kịch bản thay thế    &
  \begin{minipage}{\linewidth}
    \vskip 4pt
    \begin{enumerate}[label={\textbf{\textcolor{red}{A\arabic*}} --}, align=left, itemsep=-5pt]
      \item  Nhấn vào nút thoát.  \\
            \vspace{-1.5em}
            \begin{enumerate}[leftmargin=-5px, align=left, label=\arabic*.]
              \setcounter{enumii}{5}
              \item[]
                    \hspace{-25px} Chuỗi A1 bắt đầu ở bước 5 của kịch bản thường.
              \item Trở về giao diện chi tiết giải đấu trước đó.
              \item[]
            \end{enumerate}
      \item Không thể đăng ký tham gia giải đấu đã chọn. \\
            \vspace{-1.5em}
            \begin{enumerate}[leftmargin=-5px, align=left, label=\arabic*.]
              \setcounter{enumii}{7}
              \item[]
                    \hspace{-25px} Chuỗi A2 bắt đầu ở bước 7 của kịch bản thường.
              \item Thông báo đăng ký tham gia giải đấu không thành công.
              \item Trở về giao diện chi tiết giải đấu trước đó.  
              \item[]
            \end{enumerate}
    \end{enumerate}
    \vskip 1pt
  \end{minipage}
  \\\hline
  Kịch bản lỗi         &                                                                                              \\\hline
  Kết quả              & Đăng ký tham gia giải đấu thành công.                                                                 \\\hline
\end{longtblr}


\end{document}