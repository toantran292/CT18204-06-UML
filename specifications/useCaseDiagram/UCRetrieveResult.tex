\documentclass{article}
\usepackage[paperheight=29.7cm,paperwidth=21cm,right=2cm,left=1cm,top=2cm,bottom=2.5cm]{geometry}
\usepackage[vietnamese]{babel}
\usepackage[utf8]{inputenc}
\usepackage{array, multirow}
\usepackage{xcolor}
\usepackage{longtable}
\usepackage{lipsum}
\usepackage{enumitem}
\usepackage{tabularray}
\renewcommand{\arraystretch}{1.5} % Cell height scaling

\DefTblrTemplate{contfoot-text}{default}{Tiếp tục bên trang tiếp theo}
\DefTblrTemplate{conthead-text}{default}{(Tiếp tục)}

\begin{document}

% useCase

\begin{longtblr}[caption = {Đặc tả usecase Xem diễn biến trận đấu loại},
label = {tab:usecase1-spec},]{colspec={|l|p{.7\linewidth}|}}
\hline
\textbf{Tên usecase} & \textbf{Xem diễn biến trận đấu loại} \\\hline
Tóm tắt & Cho phép người sử dụng xem diễn biến của trận đấu loại. \\\hline
Actor & Người dùng. \\\hline
Phiên bản & 1.4 \\\hline
Chịu trách nghiệm & Trần Thái Toàn \\\hline
Ngày tạo & 10/03/2024 \\\hline
Ngày cập nhật & 26/03/2024 \\\hline
Điều kiện tiên quyết & Không có. \\\hline
Kịch bản thường &
\begin{minipage}{\linewidth}
\vskip 4pt
\begin{enumerate}
\item Người dùng truy cập vào hệ thống và chọn vào danh mục trận đấu loại.
\item Hệ thống hiển thị danh sách trận đấu. \\
\textbf{Có thể nhảy đến}\\
\textbf{\textcolor{red}{A1}} -- Người dùng chọn vào chức năng tìm kiếm trận đấu.
\item Người dùng chọn trận đấu cụ thể để xem diễn biến.
\item Hệ thống hiển thị diễn biết chi tiết của trận đấu.
\end{enumerate}
\vskip 1pt
\end{minipage}
\\\hline
Kịch bản thay thế &
\begin{minipage}{\linewidth}
\vskip 4pt
\begin{enumerate}[label={\textbf{\textcolor{red}{A\arabic*}} --}, align=left, itemsep=-5pt]
\item Người dùng chọn vào chức năng tìm kiếm. \\
\vspace{-1.5em}
\begin{enumerate}[leftmargin=-5px, align=left, label=\arabic*.]
\setcounter{enumii}{2}
\item[]
\hspace{-25px} Chuỗi A bắt đầu ở bước 2 của kịch bản thường.
\item Hệ thống hiển thị thanh tìm kiếm.
\item Người dùng nhập tên trận đấu vào thanh tìm kiếm và nhấn tìm kiếm.\\
\textbf{Có thể nhảy đến}\\
\textbf{\textcolor{red}{A2}} -- Không tìm thấy trận đấu theo tên trận đấu người dùng cung cấp.
\item Hệ thống hiển thị danh sách trận đấu theo tên trận đấu người dùng cung cấp.
\item[]
\hspace{-25px} Trở về bước 3 của kịch bản thường.
\end{enumerate}

       \item Không tìm thấy trận đấu theo tên trận đấu người dùng cung cấp. \\
            \vspace{-1.5em}
            \begin{enumerate}[leftmargin=-5px, align=left, label=\arabic*.]
              \setcounter{enumii}{4}
              \item[]
                    \hspace{-25px} Chuỗi I bắt đầu ở bước 4 của kịch bản thay thế A.
              \item Hiển thị trận đấu không tìm thấy cho người dùng.
              \item[]
                    \hspace{-25px} Trở về bước 2 của kịch bản thường.
            \end{enumerate}
    \end{enumerate}
    \vskip 1pt

\end{minipage}
\\\hline
Kịch bản lỗi & \\\hline
Kết quả & Hiện thị diễn biến trận đấu. \\\hline
\end{longtblr}
\end{document}
