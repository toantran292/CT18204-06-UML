\documentclass{article}
\usepackage[paperheight=29.7cm,paperwidth=21cm,right=2cm,left=1cm,top=2cm,bottom=2.5cm]{geometry}
\usepackage[vietnamese]{babel}
\usepackage[utf8]{inputenc}
\usepackage{array, multirow}
\usepackage{xcolor}
\usepackage{longtable}
\usepackage{lipsum} 
\usepackage{enumitem}
\usepackage{tabularray}
\renewcommand{\arraystretch}{1.5} % Cell height scaling

\DefTblrTemplate{contfoot-text}{default}{Tiếp tục bên trang tiếp theo}
\DefTblrTemplate{conthead-text}{default}{(Tiếp tục)}

\begin{document}

\begin{longtblr}[caption = {Mô tả thuộc tính của lớp VongLoai},
  label = {tab:class1-1-spec},]{
  width=1\linewidth, rowhead=1, hlines,vlines,
  colspec={X[2,c]X[1,c]X[1,c]X[1,c]X[1,c]X[1,c]X[1,c]X[3,c]},
  rows={1.5cm,m},
  row{1}={font=\bfseries,c}}
  Tên thuộc tính & Kiểu truy cập & Kiểu dữ liệu & Giá trị mặc nhiên & Kích thước & Min & Max & Diễn giải             \\
  chieuCao & private & number & 0 & & 0 & 3 &  Đơn vị: mét \\
  canNang & private & number & 0 & & 0 & 1000 & Đơn vị: kilogam \\
\end{longtblr}
  
  \begin{longtblr}[caption = {Mô tả phương thức của lớp VongLoai},
  label = {tab:class1-2-spec},]{
  width=1\linewidth, rowhead=1, hlines,vlines,
  colspec={X[3,c]X[1,c]X[3,c]X[1,c]X[1,c]X[1,c]X[2,c]X[3,c]},
  rows={1.5cm,m},
  row{1}={font=\bfseries,c}}
  Tên phương thức              & Kiểu truy cập          & Danh sách các tham số        & Kiểu dữ liệu tham số & Giá trị mặc nhiên & Kích thước & Kiểu trả về của phương thức & Diễn giải                                                                               \\
  \SetCell[r=2]{} layThongTin() & \SetCell[r=2]{} public & \SetCell[c=4]{} Có 0 tham số &                      &                   &            & \SetCell[r=2]{} VDV   & \SetCell[r=2]{} Lấy thông tin của VDV. \\
</table>
                              &                         &                &          &            &         &                             &                                                                                         \\
  \SetCell[r=2]{} layDSVDV() & \SetCell[r=2]{} public & \SetCell[c=4]{} Có 0 tham số &                      &                   &            & \SetCell[r=2]{}List < VDV >  & \SetCell[r=2]{} Lấy danh sách tất cả các VDV \\
  </table>
                                &                         &                &          &       &            &                             &                                                          \\                              
  \SetCell[r=7]{} yeuCauSuaThongTin() & \SetCell[r=7]{} public & \SetCell[c=4]{} Có 6 tham số &                      &                   &            & \SetCell[r=7]{}boolean   & \SetCell[r=7]{} VDV gửi yêu cầu chỉnh sửa thông tin của mình \\
  </table>
                          &                       & hoTen                & String         & Null           & 30           &                             &                               \\ 
                          &                       & gioiTinh        & boolean       & Null           &            &                             &                               \\ 
                          &                       &    ngaySinh     & Date         & Null           &            &                             &                               \\ 
                          &                       & std             & String          & Null           & 10           &                             &                               \\ 
                          &                       &    canNang      & number          & Null           &            &                             &                               \\ 
                          &                       &    chieuCao      & number         & Null           &            &                             &                               \\ 

  \SetCell[r=2]{} inDSVDV() & \SetCell[r=2]{} public & \SetCell[c=4]{} Có 0 tham số &                      &                   &            & \SetCell[r=2]{}booolean  & \SetCell[r=2]{} In danh sách tất cả vận động viên \\
  </table>
                          &                       &                 &          &            &            &                             &                               \\                                 

\end{longtblr}
  \end{document}