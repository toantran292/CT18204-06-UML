\documentclass{article}
\usepackage[paperheight=29.7cm,paperwidth=21cm,right=2cm,left=1cm,top=2cm,bottom=2.5cm]{geometry}
\usepackage[vietnamese]{babel}
\usepackage[utf8]{inputenc}
\usepackage{array, multirow}
\usepackage{xcolor}
\usepackage{longtable}
\usepackage{lipsum} 
\usepackage{enumitem}
\usepackage{tabularray}
\renewcommand{\arraystretch}{1.5} % Cell height scaling

\DefTblrTemplate{contfoot-text}{default}{Tiếp tục bên trang tiếp theo}
\DefTblrTemplate{conthead-text}{default}{(Tiếp tục)}

\begin{document}

\begin{longtblr}[caption = {Mô tả thuộc tính của lớp VaiTro},
  label = {tab:class1-1-spec},]{
  width=1\linewidth, rowhead=1, hlines,vlines,
  colspec={X[2,c]X[1,c]X[1,c]X[1,c]X[1,c]X[1,c]X[1,c]X[3,c]},
  rows={1.5cm,m},
  row{1}={font=\bfseries,c}}
  Tên thuộc tính & Kiểu truy cập & Kiểu dữ liệu & Giá trị mặc nhiên & Kích thước & Min & Max & Diễn giải             \\
  idVaiTro & private & string & null & 8&  &  & id vai trò. \\
  tenVaiTro & private & string & null & 30&  &  & Tên vai trò
\end{longtblr}
  
  \begin{longtblr}[caption = {Mô tả phương thức của lớp VaiTro},
  label = {tab:class1-2-spec},]{
  width=1\linewidth, rowhead=1, hlines,vlines,
  colspec={X[3,c]X[1,c]X[3,c]X[1,c]X[1,c]X[1,c]X[2,c]X[3,c]},
  rows={1.5cm,m},
  row{1}={font=\bfseries,c}}
  Tên phương thức              & Kiểu truy cập          & Danh sách các tham số        & Kiểu dữ liệu tham số & Giá trị mặc nhiên & Kích thước & Kiểu trả về của phương thức & Diễn giải                                                                               \\
  \SetCell[r=2]{} suaVaiTro() & \SetCell[r=2]{} public & \SetCell[c=4]{} Có 1 tham số &                      &                   &            & \SetCell[r=2]{}boolean   & \SetCell[r=2]{} Sửa thông tin của 1 vai trò. \\
</table>
                              &                         & tenVaiTro               & String         & Null           & 8           &                             &                                                                                         \\
  \SetCell[r=3]{} thietLapVaiTro() & \SetCell[r=3]{} public & \SetCell[c=4]{} Có 2 tham số &                      &                   &            & \SetCell[r=3]{}boolean   & \SetCell[r=3]{} Thiết lập vai trò của vận động viên trong đội hình mẫu. Trả về true nếu thiết lập thành công,ngược lại là false \\
  </table>
                                &                         & idVDV               & String         & Null           & 8           &                             &                                                                                         \\                              
                                &                         & tenVaiTro               & String         & Null        & 30            &                             &                                                                                         \\                              
\end{longtblr}
  \end{document}