\documentclass{article}
\usepackage[paperheight=29.7cm,paperwidth=21cm,right=2cm,left=1cm,top=2cm,bottom=2.5cm]{geometry}
\usepackage[vietnamese]{babel}
\usepackage[utf8]{inputenc}
\usepackage{array, multirow}
\usepackage{xcolor}
\usepackage{longtable}
\usepackage{lipsum}
\usepackage{enumitem}
\usepackage{tabularray}
\renewcommand{\arraystretch}{1.5} % Cell height scaling

\DefTblrTemplate{contfoot-text}{default}{Tiếp tục bên trang tiếp theo}
\DefTblrTemplate{conthead-text}{default}{(Tiếp tục)}

\begin{document}

% Class
\begin{longtblr}[caption = {Mô tả thuộc tính của lớp Loi},
label = {tab:class1-1-spec},]{
width=1\linewidth, rowhead=1, hlines,vlines,
colspec={X[2,c]X[1,c]X[1,c]X[1,c]X[1,c]X[1,c]X[1,c]X[3,c]},
rows={1.5cm,m},
row{1}={font=\bfseries,c}}
Tên thuộc tính & Kiểu truy cập & Kiểu dữ liệu & Giá trị mặc nhiên & Kích thước & Min & Max & Diễn giải \\
phamLoi & Public & String & Null & 200 & & & Mô tả lỗi mà vận động viên vi phạm trong một trận đấu.\\
\end{longtblr}

\begin{longtblr}[caption = {Mô tả phương thức của lớp Loi},
label = {tab:class1-2-spec},]{
width=1\linewidth, rowhead=1, hlines,vlines,
colspec={X[3,c]X[1,c]X[3,c]X[1,c]X[1,c]X[1,c]X[2,c]X[3,c]},
rows={1.5cm,m},
row{1}={font=\bfseries,c}}
Tên phương thức & Kiểu truy cập & Danh sách các tham số & Kiểu dữ liệu tham số & Giá trị mặc nhiên & Kích thước & Kiểu trả về của phương thức & Diễn giải \\
\SetCell[r=1]{} layDSLoi() & \SetCell[r=1]{} Public & \SetCell[c=4]{} Không có tham số & & & & \SetCell[r=1]{} List<Loi> & \SetCell[r=1]{} Lấy danh sách lỗi vi phạm. \\

\SetCell[r=2]{} taoLoi() & \SetCell[r=2]{} Public & \SetCell[c=4]{} Có 1 tham số,& & & & \SetCell[r=2]{} Boolean & \SetCell[r=2]{} Tạo lỗi vi phạm của vận động viên trong trận đấu. \\
& & loi & Loi & & & & \\
\SetCell[r=1]{} thongKeLoi() & \SetCell[r=1]{} Public & \SetCell[c=4]{} Không có tham số & & & & \SetCell[r=1]{} & \SetCell[r=1]{} Thống kê lỗi vi phạm. \\
\end{longtblr}
\end{document}
